\renewcommand{\arraystretch}{1.1}
\setlength{\tabcolsep}{8pt}
Consider a linear programming problem in standard form, described in terms of the following tableau:
\[
\begin{tabular}{c|c|ccccccc}
  &   & $x_1$ & $x_2$ & $x_3$ & $x_4$ & $x_5$ & $x_6$ & $x_7$\\
\hline
  & $0$  & $0$ & $0$ & $0$ & $\delta$ & 3 & $\gamma$ & $\xi$ \\
\hline
$x_2=$   & $\beta$ & 0 & 1 & 0 & $\alpha$ & 1 & 0      &  3 \\
$x_3=$   & 2       & 0 & 0 & 1 & -2\,\,       & 2 & $\eta$ & -1\,\, \\
$x_1=$   & 3       & 1 & 0 & 0 &  0       &-1\,\, & 2      &  1
  \end{tabular}
\]
The entries $\alpha$, $\beta$, $\gamma$, $\delta$, $\eta$ and $\xi$ in the tableau are unknown parameters. Furthermore, let $\mathbf{B}$ be the basis matrix corresponding to having $x_2$, $x_3$, and $x_1$ (in that order) be the basic variables. For each one of the following statements, find the ranges of values of the various parameters that will make the statement to be true.
\begin{itemize}
\item[(a)] Phase II of the Simplex method can be applied using this as an initial tableau.
%\item[(b)] The first row in the present tableau (below the row with the reduced costs) indicates that the problem is infeasible.
\item[(b)] The corresponding basic solution is feasible, but we do not have an optimal basis.
\item[(c)] The corresponding basic solution is feasible and the first Simplex iteration indicates that the optimal cost is $-\infty$.
\item[(d)] The corresponding basic solution is feasible, $x_6$ is candidate for entering the basis, and when $x_6$ is the entering variable, $x_3$ leaves the basis.  
\item[(e)] The corresponding basic solution is feasible, $x_7$ is candidate for entering the basis, but if it does, the objective value remains unchanged.
\end{itemize}